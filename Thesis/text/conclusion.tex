
\chapter{Conclusion and Future Work}

In this thesis, at the beginning of the project, we addressed the issue of trajectory simplification and its realization in an online framework, so that it can be used as a node process in a streaming model. First and foremost, the state-of-the-art was analyzed in order to understand the various technologies and algorithms that address the problem. Then we selected a solution that corresponded to our expectations and analyzed it in order to understand it and to identify any possible challenges or problems linked to the algorithm. Next, we implemented the solution in a C code language that would enable it to be implemented in MobilityDB and also to be used as part of real-time processing by the technologies studied above. Next, we compared offline and online solutions to get an idea of the gap in simplification caused by streaming. Our observation is that the loss of accuracy between online and offline configuration is very small.  Lastly, there was a comparison with existing simplification algorithms on MobilityDB to get an idea of how the algorithm ranks against existing algorithms. SQUISH-E is a good candidate for solving the real-time simplification problem while maintaining a certain level of accuracy and performance. \\\\
First of all, we can look at different ways of comparing the accuracy of different trajectories. It's important to remember that the measurements and metrics used can be improved, and there are a huge number of metrics that could be useful, such as the variation in speed patterns between trajectories, or the volume of differences between two trajectories, as LIP does \cite{4438678}. It would therefore be a good idea to run and compare algorithms using different metrics to get an idea of their performance according to other criteria. Next we can note that the SQUISH-E algorithm has been modified during the process of this work to obtain a fully online algorithm. There are several possible improvements, starting with the choice of error function: in our case, SQUISH-E calculates its error distance using SED (synchronized euclidian distance), but it would be worth testing other error functions, or even inventing new ones, to see if it is possible to vary this parameter for use cases where other metrics become important, such as speed. As this work has not made any comparison between online algorithms, it is essential to compare online algorithms with each other and also to test them in real-life situations in order to obtain the perception of the reality gap between theory and practice. Finally, we'll see if we can improve implementation to reduce possible memory leaks, or code quality to make code more maintainable. All these ideas are part of the work produced and can be investigated.
