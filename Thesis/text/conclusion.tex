
\chapter{Conclusion and Future Work}

In this thesis, we tackled the problem of trajectory simplification and its implementation in an online framework, suitable for integration into a streaming model. Initially, we conducted an extensive review of the state-of-the-art technologies and algorithms addressing trajectory simplification. This helped us understand the strengths and weaknesses of various approaches. After careful consideration, we selected the SQUISH-E algorithm for its potential to meet our needs and conducted an in-depth analysis to identify any challenges or limitations inherent to the algorithm.

We then implemented the SQUISH-E algorithm in C, ensuring compatibility with MobilityDB and enabling its use in real-time processing systems. The implementation phase was followed by a comparative analysis between offline and online versions of the algorithm to quantify the impact of streaming on simplification accuracy. Our findings indicate that the loss of accuracy in the online configuration is minimal, demonstrating the robustness of SQUISH-E in real-time applications.

Further, we compared SQUISH-E with other existing simplification algorithms available in MobilityDB to evaluate its performance. The results show that SQUISH-E is a strong candidate for real-time simplification tasks, maintaining a commendable balance between accuracy and performance. MinDist, another algorithm evaluated during this study, also exhibited high levels of accuracy and performance. Given its iterative nature, MinDist shows promise for adaptation to an online setting.

Our project achieved several key outcomes:
\begin{itemize}
    \item We successfully implemented an online trajectory simplification algorithm in C for use with MobilityDB.
    \item We demonstrated that the accuracy loss from an online configuration is minimal, supporting the feasibility of real-time trajectory simplification.
    \item Comparative analysis with existing algorithms highlighted SQUISH-E's competitive performance, making it a viable option for real-time applications.
    \item We identified MinDist as another potential algorithm for online trajectory simplification, suggesting directions for future research.
\end{itemize}

\subsection*{Future Work}

Future research can explore various dimensions to build upon this work. Here are some detailed suggestions for future directions:

\paragraph{Diverse Metrics for Accuracy Assessment}
While our study used specific metrics to compare the accuracy of trajectory simplifications, there are many other metrics that could provide additional insights. For instance, examining variations in speed patterns between trajectories or the volumetric differences between two trajectories, as in the LIP method \cite{4438678}, could offer new perspectives on algorithm performance. Future work should consider running and comparing algorithms using a broader set of metrics to understand their strengths and weaknesses in different scenarios.

\paragraph{Enhancements to SQUISH-E Algorithm}
During the course of this research, we modified the SQUISH-E algorithm to function fully online. There are several potential improvements:
\begin{itemize}
    \item \textbf{Alternative Error Functions:} Currently, SQUISH-E uses the Synchronized Euclidean Distance (SED) as its error metric. It would be beneficial to experiment with other error functions or develop new ones tailored to specific use cases, such as speed or direction changes.
    \item \textbf{Iterative Variations:} Applying similar modifications to MinDist and MaxDist to create online variants could yield competitive alternatives for real-time simplification.
\end{itemize}

\paragraph{Comparative Analysis of Online Algorithms}
This thesis primarily focused on comparing online algorithms with their offline counterparts. Future studies should conduct extensive comparisons between different online algorithms to determine their relative performance. Additionally, testing these algorithms in real-world scenarios will provide valuable insights into the practical implications of theoretical results.

\paragraph{Implementation Improvements}
Optimizing the implementation to minimize potential memory leaks and enhance code quality is crucial. Future work should focus on:
\begin{itemize}
    \item \textbf{Memory Management:} Ensuring efficient memory usage to prevent leaks and optimize performance.
    \item \textbf{Code Quality:} Refactoring the code to improve maintainability and ease of updates or modifications.
\end{itemize}

\paragraph{Real-World Application and Testing}
Deploying the algorithms in real-life applications and collecting empirical data on their performance will bridge the gap between theory and practice. This will also help in identifying unforeseen challenges and areas for improvement.

By addressing these areas, future researchers can extend the work presented in this thesis, contributing to the development of more robust and efficient online trajectory simplification algorithms.

%In this thesis, at the beginning of the project, we addressed the issue of trajectory simplification and its realization in an online framework, so that it can be used as a node process in a streaming model. First and foremost, the state-of-the-art was analyzed in order to understand the various technologies and algorithms that address the problem. Then we selected a solution that corresponded to our expectations and analyzed it in order to understand it and to identify any possible challenges or problems linked to the algorithm. Next, we implemented the solution in a C code language that would enable it to be implemented in MobilityDB and also to be used as part of real-time processing by the technologies studied above. Next, we compared offline and online solutions to get an idea of the gap in simplification caused by streaming. Our observation is that the loss of accuracy between online and offline configuration is very small.  Lastly, there was a comparison with existing simplification algorithms on MobilityDB to get an idea of how the algorithm ranks against existing algorithms. SQUISH-E is a good candidate for solving the real-time simplification problem while maintaining a certain level of accuracy and performance. MinDist also shown high level of accuracy and performance and could be improved also be implemented in an online setting since it is an iterative algorithm.  \\\\

%First of all, we can look at different ways of comparing the accuracy of different trajectories. It's important to remember that the measurements and metrics used can be improved, and there are a huge number of metrics that could be useful, such as the variation in speed patterns between trajectories, or the volume of differences between two trajectories, as LIP does \cite{4438678}. It would therefore be a good idea to run and compare algorithms using different metrics to get an idea of their performance according to other criteria. Next we can note that the SQUISH-E algorithm has been modified during the process of this work to obtain a fully online algorithm. There are several possible improvements, starting with the choice of error function: in our case, SQUISH-E calculates its error distance using SED (synchronized euclidian distance), but it would be worth testing other error functions, or even inventing new ones, to see if it is possible to vary this parameter for use cases where other metrics become important, such as speed. In the same manner apply this variation on MinDist and MaxDist to have a concurent online algorithm. As this work has not made enough comparisons between online algorithms, it is essential to compare online algorithms with each other and also to test them in real-life situations in order to obtain the perception of the reality gap between theory and practice. Finally, we'll see if we can improve implementation to reduce possible memory leaks, or code quality to make code more maintainable. All these ideas are part of the work produced and can be investigated. 
